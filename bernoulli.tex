% -*- mode: latex; eval: (flyspell-mode 1); ispell-local-dictionary: "american"; TeX-master: t; -*-

\documentclass[12pt]{article}



\usepackage{hyperref,amsthm,amsmath,amsfonts} 

\newcommand{\bZ}{\mathbb{Z}} 
\newcommand{\bR}{\mathbb{R}} 
\newcommand{\bN}{\mathbb{N}}
\newcommand{\abs}[1]{\left| #1 \right|}
\newcommand{\gtint}[1]{\left] #1, \infty \right[}
\newcommand{\geint}[1]{\left[ #1, \infty \right[}
\newcommand{\ltint}[1]{\left]- \infty, #1 \right[}
\newcommand{\leint}[1]{\left]- \infty, #1 \right]}

\newcommand{\ttx}{\mathtt{x}}
\newcommand{\tty}{\mathtt{y}}

\newtheorem{theorem}{Theorem}
% \newtheorem{proposition}{Proposition}
\newtheorem{lemma}{Lemma}

\begin{document}

\sloppy

Let $\bR$ denote the field of real numbers and let $\bN$ denote the set of all \emph{non-negative} integers.
The aim of this paper is to study 
$$
B = \left\{ (n, x) \in \bN \times \bR : {(1 + x)}^n \ge 1 + n x \right\} \, . 
$$
Let $\ttx$ denote an indeterminate over~$\bR$.
For each $n \in \bN$, let $f_n$ be the polynomial with integer coefficients defined by:
\begin{equation} \label{eq:def-fn}
f_n(\ttx) = {(1 + \ttx)}^n - 1 - n \ttx  \,.
\end{equation}
The fundamental property of the $f_n$'s is  
$$
B = \left\{ (n, x) \in \bN \times \bR : f_n(x) \ge 0 \right\} \, . 
$$
% The fundamental property of our polynomials is
% that the following equivalence holds true for every $n \in \bN$ and every $x \in \bR$:
% $$
% f_n(x) \ge 0 \iff {(1 + x)}^n \ge 1 + n x \,.
% $$ 
%The latter inequality is called \emph{Bernoulli's inequality}.
%Hence, studying Bernoulli's inequality is the same as studying the signs of the $f_n$'s.
For each $n \in \bN$, the first and second derivatives of $f_n$ are given by
\begin{equation} \label{eq:deriv-fn} 
  f_n'(\ttx)  = n \left( {(1 + \ttx)}^{n - 1} -  1 \right) 
\end{equation}
and
\begin{equation} \label{eq:deriv-second-fn}
f_n''(\ttx)  = n (n - 1) {(1 + \ttx)}^{n - 2} \, .
\end{equation}

\begin{theorem} \label{thm:root-mult}
  For each $n \in \bN + 2$,
  $0$ is a double root of $f_n$ and $0$ the only non-simple (complex) root of $f_n$.
\end{theorem}

\begin{proof}
  By letting $\ttx = 0$ in Equations \eqref{eq:def-fn}, \eqref{eq:deriv-fn}, and \eqref{eq:deriv-second-fn},
  we obtain 
  $$
  f_n(0) = f_n'(0) = 0 \ne n (n - 1) = f_n''(0) \, ,
  $$
  whence $0$ is a double root of $f_n$.
  In addition, $f_n$ and its derivative satisfy 
   $$
   (1 + \ttx) f_n'(\ttx) - n f_n(\ttx) = n (n - 1) \ttx \, ,
   $$
   so for every complex number $z$, $f_n(z) = f_n'(z) = 0$ implies $z = 0$.
   It follows that $0$ is the only non-simple root of $f_n$.
 \end{proof}
 
 Theorem~\ref{thm:root-mult} ensures that
for every $n \in \bN + 2$ and every $x \in \bR$, 
the graph of $f_n$ \emph{crosses} the $\ttx$-axis at $\ttx = x$ if, and only if, $x$ is a non-zero root of $f_n$
(the graph of $f_n$ \emph{touches} the $\ttx$-axis at $\ttx = 0$ but does not cross).
A simple examination of Equation~\eqref{eq:deriv-fn} is enough to determine the sign of $f_n'$ for every $n \in \bN + 2$:

\begin{theorem} \label{thm:variation}
    For each $n \in 2 \bN + 2$,
    $f_n'$ is
    negative on $\ltint{0}$ and
    positive on $\gtint{0}$.
   For each $n \in 2 \bN + 3$, $f_n'$ is
   positive on $\ltint{- 2}$,
   negative on $\left]- 2, 0 \right[$, and
   positive on $\gtint{0}$.
 \end{theorem}

 
For instance, we have 
$f_0(\ttx) = f_1(\ttx) = 0$,
$f_2(\ttx) = \ttx^2$,
\begin{equation} \label{eq:f3-t3}
f_3(\ttx) = \ttx^3 + 3 \ttx^2 = \ttx^2 (\ttx + 3) \,.
\end{equation}
Note that
$$\bN + 2 = \bN \setminus \{ 0, 1 \} = (2 \bN + 2) \cup (2 \bN + 3) \, .$$
The following two theorems are immediate corollaries of Theorem~\ref{thm:variation}:

 \begin{theorem}%[Bernoulli's inequality]
   \label{thm:Bernoulli}
   For every $n \in \bN + 2$ and every $x \in \bR \setminus \{ 0 \}$ such that $n \in 2 \bN$ or $x \ge - 2$,
   $f_n(x)$ is positive.
   For every $n \in \bN$ and every $x \in \geint{-2}$, $f_n(x)$ is non-negative.
 \end{theorem}

 Theorem~\ref{thm:Bernoulli} shows  
 
\begin{theorem} \label{thm:increasing-fn}
  For each $x \in \left]- 2, 0 \right[ \cup \gtint{0}$,
  the sequence $\left( f_n(x) \right)_{n \in \bN + 1}$ is increasing.
\end{theorem} 

\begin{proof}
  Let $n \in \bN + 1$.
 Straightforward computations yield  
 $$
 f_{n+ 1} (x) - f_n(x) = \frac{1}{n + 1}  \ttx f_{n + 1}'(\ttx) 
 $$
 and
 Theorem~\ref{thm:variation} ensures that $\ttx f_{n + 1}(\ttx)$ is positive on $\left]- 2, 0 \right[ \cup \gtint{0}$.
\end{proof} 


 \begin{proof}
   First,
   if $x = - 2$ then 
   we have $f_n(x) = {(- 1)}^n - 1 + 2n \ge 2 n - 2 > 0$.
   Second,
   if $x > - 2$ then Theorem~\ref{thm:increasing-fn} ensures $f_n(x) > f_1(x) =  0$.
   Third and last, assume $n \in 2 \bN + 2$ and $x < - 2$.
   Then, both ${(1 + x)}^n$ and $- 1 - n x$ are positive,
   and thus their sum, which is equal to $f_n(x)$, is also positive.
 \end{proof}
 


 \begin{proof}
   The desired result is an immediate corollary of Theorem~\ref{thm:Bernoulli}.
 \end{proof}

 Theorems \ref{thm:Bernoulli} and \ref{thm:Bernoulli-non-strict} can also be obtained as corollaries of:
 
  \begin{theorem} \label{thm:variation}
    For each $n \in 2 \bN + 2$,
    $f_n$ is
    decreasing on $\leint{0}$ and
    increasing on $\geint{0}$.
   For each $n \in 2 \bN + 3$, $f_n$ is
   increasing on $\leint{- 2}$,
   decreasing on $\left]- 2, 0 \right[$, and
   increasing on $\geint{0}$.
\end{theorem}

\begin{theorem}
  Let $x$, $y \in \bR$ be such that $x \ne y$.
   For each $n \in 2 \bN + 2$, 
   $$
  \frac{x^n - y^n}{x - y} = \sum_{k = 0}^{n - 1} \ttx^{n - 1 - k}\tty^k 
   $$
   
   $$
   f_n(x - 1) - f_n(y - 1) = {1 + x}^n - 
    = 
   {(1 + \ttx)}^n - {(1 + \tty)}^n - n (\ttx - \tty)
   $$
   $$
   \sum_{k = 0}^{n - 1} \left(  {(1 + \ttx)}^{n - 1 - k}{(1 + \tty)}^k - 1  \right)
   $$
 \end{theorem}

 

 
 \section{Calculus stuff}

%  For each $n \in \bN + 2$,
%  straightforward computations yield
% \begin{equation} \label{eq:deriv-fn} 
%   f_n'(\ttx)  = n \left( {(1 + \ttx)}^{n - 1} -  1 \right) \, , 
%   \end{equation}
%  so the sign of $f_n'$ is easy to determine:

Theorem~\ref{thm:Bernoulli} is an immediate corollary of Theorem~\ref{thm:variation}.
 $$
  f_n''(\ttx)  = n (n - 1) {(1 + \ttx)}^{n - 2} \,.
  $$


 \begin{theorem} \label{thm:variation-even}
   For each $n \in 2 \bN + 2$,
   $f_n$ is
   decreasing on $\leint{0}$ and
   increasing on $\geint{0}$.
 \end{theorem}

 \begin{proof}
   Since $n - 1$ is an odd positive integer,
   Equation~\eqref{eq:deriv-fn} shows that
   $f'_n$ is increasing on $\bR$,
   and subsequently, that
   $0$ is the only real root of $f_n'$.
   Therefore,
   $f'_n$ is negative on $\ltint{0}$
   and 
   positive on $\gtint{0}$.
   The desired result follows.
 \end{proof}



 \begin{proof}
   Since $n - 1$ is an even positive integer,
   the function from $\bR$ to itself associated with the monomial $\ttx^{n - 1}$ is
   decreasing on $\leint{0}$ and
   increasing on $\geint{0}$.
   Hence, 
   Equation~\eqref{eq:deriv-fn} shows: 
   $f_n'$ is decreasing on $\leint{- 1}$,
   $f_n'$ vanishes at $- 2$,
   $f_n'$ is increasing on $\geint{- 1}$, and
   $f_n'$ vanishes at~$0$.
   Therefore, $f_n'$ is positive on $\ltint{- 2}$,
   negative on $\left]- 2, 0 \right[$, and
   positive on $\gtint{0}$.
   The desired result follows.
 \end{proof}
 
 

 \section{Previous results} 
 
 \begin{theorem} \label{thm:tipping-point}
   For each odd  $n \in 2 \bN + 3$,
   there exists $t_n \in \ltint{- 2}$ such that
 \begin{equation} \label{eq:def-tn}
 \left\{ x \in \bR : f_n(x) \ge 0 \right\}
 =
 \geint{t_n} \,. 
 \end{equation} 
 \end{theorem} 

 \begin{proof}
   By Theorem~\ref{thm:variation-odd},  $f_n$ is decreasing on $[- 2, 0]$,
   whence   $f_n(- 2) > f_n(0) = 0$.
   Moreover, $f_n(x)$ is asymptotically equivalent to $x^n$ as $x$ approaches $\pm \infty$,
   and thus $f_n(x)$ approaches $- \infty$ as $x$ approaches $- \infty$.
   Therefore, the intermediate value theorem ensures that there exists $t_n \in \ltint{- 2}$ such that $f_n(t_n) = 0$.
   It follows from Theorem~\ref{thm:variation-odd} that Equation~\eqref{eq:def-tn} holds true.
 \end{proof}

 For each $n \in 2 \bN + 3$,
 let $t_n$ be as in Theorem~\ref{thm:tipping-point}:
 $t_n$ is a \emph{tipping point} for Bernoulli's inequality.
 Equation~\eqref{eq:f3-t3} shows that $t_3 = - 3$. 
 Straightforward computations yield the following recurrence relation which is used as a lemma:
 \begin{equation} \label{eq:fn+2-fn}
   f_{n + 2} (\ttx) = {(1 + \ttx)}^2 f_n(\ttx) + n \ttx^2 \left(\ttx + 2 + \frac{1}{n} \right)
 \end{equation}
 for every $n \in \bN + 1$.
 
 \begin{theorem} \label{thm:HP-upper}
   For each $n \in 2 \bN + 3$, $t_n$ is not greater than $- 2 - {(n - 2)}^{-1}$.
 \end{theorem}

 \begin{proof}
   Put $u_n = - 2 - n^{-1}$ for every $n \in \bN + 1$.
   Our task it to prove $t_n \le u_{n - 2}$ for every $n \in 2 \bN + 3$.
   We proceed by induction on~$n$.
   The assertion holds true for $n = 3$ because $t_3 = u_1 = - 3$.
   Let us now deal with the inductive step.
   Let $n \in 2 \bN + 3$ be such that $t_n \le u_{n - 2}$.
   Since $u_{n - 2}$ not greater than $u_n$, $t_n$ is not greater than $u_n$, or, equivalently, $f_n(u_n)$ is non-negative.
   Therefore,  
   by letting $\ttx = u_n$ in Equation~\eqref{eq:fn+2-fn}, we obtain (note that the rightmost term vanishes)
   $f_{n + 2}(u_n) = {( 1 + u_n )}^2 f_n(u_n) \ge 0$.
   It follows $t_{n + 2} \le u_n$, as desired.
 \end{proof}
 
 
 
\begin{theorem} \label{thm:HP-tn-increase} 
  The sequence $\left( t_n \right)_{n \in 2 \bN + 3}$ is increasing.
 \end{theorem} 

 \begin{proof}
 Let $n \in 2 \bN + 3$.
 By letting $\ttx = t_n$ in Equation~\eqref{eq:fn+2-fn}, we obtain
 $$ 
f_{n + 2} (t_n) = n t_n^2 \left(t_n - 2 - \frac{1}{n} \right) \, .
$$
Besides, Theorem~\ref{thm:HP-upper} ensures that the right-hand side of the latter equality is negative.
It follows $f_{n + 2} (t_n) < 0$ or, equivalently, $t_n < t_{n + 2}$.
\end{proof}

For each $n \in 2 \bN + 5$, $t_n$ lies in $\left]- 3, - 2 \right[$ and the coefficients of $f_n$ are all integers,
so the integral root theorem ensures that $t_n$ is irrational.
 In the literature \cite{MitrinovicCNIA, MitrinovicAI, MitrinovicP93, MondP94},
 both Theorems \ref{thm:HP-upper} and \ref{thm:HP-tn-increase} are attributed to N.~Had\v{z}iivanov and I.~Prodanov.

\section{Our contribution}

\begin{theorem}
  Let $a \in \gtint{0}$ and let $b \in \bR$.
 As $x$ approaches $\infty$, 
 ${(a x + b)}^{1 / x} - 1$
 and
 $x^{-1} \ln x$ are asymptotically equivalent.
\end{theorem}

\begin{proof}
  
  $$
  {(a x + b)}^{1 / x} - 1 = \exp \left( \frac{\ln (a x + b)}{x} \right) - 1 \ge   \frac{\ln (a x + b)}{x} 
  $$
  
  $$
  \exp(y) = 1 + y + \frac{1}{2} y^2 + O(y^3)  
  $$

  $$
  \ln(1 + y) = O(y) 
  $$

  $$
  \ln(a x + b) = \ln (a x) + \ln \left(1 +  \frac{b}{a x} \right) = \ln(a x) + O \left( \frac{1} {x} \right)
  $$

  $$
   \frac{\ln(ax + b)}{x} = \frac{\ln(a x)}{x} + O \left( \frac{1}{x^2} \right)
  $$
  $$
  {(a x + b)}^{1 / x} = \exp \left( \frac{\ln( a x + b)}{x}  \right) = 1 + 
  $$
\end{proof} 

\begin{proof}
  Put $\epsilon_n = - t_n - 2$: $t_n = - 2 - \epsilon_n$.
  Equality
  $$
  {(1 + t_n)}^n = 1 + n t_n 
  $$
  $$
   {(1 + \epsilon_n)}^n  = - 1 + 2 n + n \epsilon_n 
   $$

   $$
   2 n - 1 \le {(1 + \epsilon_n)}^n  \le  2 n 
   $$

   $$
   \sqrt[n]{2 n - 1} - 1 \le \epsilon_n  \le \sqrt[n]{2 n} - 1
   $$

   $$
   \exp(x) - 1 \ge x
   $$
   $$
   \sqrt[n]{2n - 1} - 1 \ge \frac{\ln(2n - 1)}{n} \sim \frac{\ln(2n)}{n} 
   $$

   $$
   \exp(x) - 1 \sim  x
   $$
   
   $$
   \sqrt[n]{2n \pm 1} - 1 \sim \frac{\ln(2n \pm 1)}{n} \sim \frac{\ln(2n)}{n} 
   $$
   
\end{proof}
Let $n \in \bN$.
Straightforward computations yield
$$
f_n(- 2 - \ttx) = {(- 1)}^n {(1 + \ttx)}^n - 1 + 2n + n \ttx \, , 
$$
whence 
\begin{equation} \label{eq:fn-2-x-odd}
  n \in 2 \bN + 1
  \implies 
 f_n(- 2 - \ttx) = 2n - 2 - f_n(\ttx)   \, .
\end{equation} 
 
\begin{theorem} \label{thm:lower-sqrt}
  For every $n \in 2 \bN + 3$, $t_n$ is not less than $- 2 - 2 / \sqrt{n}$.
\end{theorem}

\begin{proof}
  Let $n \in 2 \bN + 3$.
 Put $\epsilon_n = 2 / \sqrt{n}$.
  Since $\epsilon_n \ge 0$,
  the binomial theorem yields
  $$
  f_n(\epsilon_n ) \ge \frac{1}{2} (n - 1) n \epsilon_n^2 = 2 n - 2 \,.
  $$
  It follows $f_n(- 2 - \epsilon_n) \le 0$ by Equation~\eqref{eq:fn-2-x-odd}, whence $t_n \ge - 2 - \epsilon_n$.
\end{proof} 

Note that Theorem~\ref{thm:lower-sqrt} ensures that $t_n$ approaches $- 2$ as $n$ approaches~$\infty$.


 \bibliographystyle{plain}
  \bibliography{bernoulli}

\end{document}

\begin{theorem}
  For each $n \in 2 \bN + 5$, $t_n$ is less than $- 2 - {(n - 2)}^{-1}$.
\end{theorem}

\begin{proof}
  Let $n \in 2 \bN + 5$.
  By letting $\ttx = - 2 - n^{-1}$ in Equation~\eqref{eq:fn+2-fn} we obtain
\end{proof}
 \begin{proof}
   Let $n \in \bN + 1$.
   % Our task is to prove $f_{n + 2}(- 2 - n^{-1}) \ge 0$.
   Let us first check $f_n(- 2 - n^{- 1}) \ge 0$.
   Theorem~\ref{thm:Bernoulli} ensures $f_n(- {(n + 1)}^{-1}) \ge 0$, whence
   $$
   \left( 1 - \frac{1}{n + 1} \right)^n \ge 1 - \frac{n}{n + 1} = \frac{1}{n + 1} \,.
   $$
   Since 
   $$
   1 + \frac{1}{n} = \left( 1 - \frac{1}{n + 1} \right)^{-1} \,,
   $$
   it follows 
   $$
  \left( 1 + \frac{1}{n} \right)^n = \left( 1 - \frac{1}{n + 1} \right)^{-n}  \ge - (n + 1)  
  $$
  (a proof that ${(1 + n^{-1} )}^n$ is less than $e$ requires a little more work \cite{Wiener85}).
  and subsequently,
  $$
  f_n \left(- 2 - \frac{1}{n} \right) = {(- 1)}^n \left( 1 + \frac{1}{n} \right)^n + 2 n \ge - (n + 1) + 2 n = n - 1 \ge 0 \,.
  $$
  Besides, straightforward computations yield 
   \begin{equation} \label{eq:fn+2-fn}
   f_{n + 2} (\ttx) = {(1 + \ttx)}^2 f_n(\ttx) + n \ttx^2 \left(\ttx - 2 - \frac{1}{n} \right) \, ,
 \end{equation}
 whence
 $$
 f_{n + 2}\left(- 2 - \frac{1}{n} \right) = \left( 1 + \frac{1}{n} \right)^2 f_n\left( - 2 - \frac{1}{n} \right) \ge 0 \, . 
 $$
 \end{proof} 
   Then, the previous discussion ensures that $f_{n / 2}$ is non-negative on $\geint{-1}$.
   In particular, $f_{n / 2}(y) $ is non-negative for $y = {(1 + x)}^2 - 1$
   (note that $y = 0$ is equivalent to $x = - 2$).
   Besides, straightforward computations yield
   $$
   f_{2 m}(\ttx) = f_m({(1 + \ttx)}^2 - 1) + m \ttx^2 
   $$
   for every $m \in \bN$.
   It follows
   $$
   f_n(x) = f_{n / 2}( y ) + \frac{n}{2} y^2 \ge \frac{n}{2} x^2 > 0
   $$
and 
$$
f_4(\ttx) = \ttx^4 + 4 \ttx^3 + 6 \ttx = \ttx^2 \left( {(\ttx + 2)}^2 + 4 \right)  \, .
$$
 The following identities are not mentioned anywhere else in the paper but may be of interest to the reader:
 $$
 f_n (\ttx)
 = \ttx^2 \sum_{k = 2}^n \binom{n}{k} \ttx^{k - 2}
 = \ttx^2 \sum_{k = 1}^{n - 1} \sum_{j = 0}^{k - 1} {(1 + \ttx)}^j \, .  
 $$
for every $x \in \bR$, $f_n(x) \ge 0$ is equivalent to ${(1 + x)}^n \ge 1 + n x$. 