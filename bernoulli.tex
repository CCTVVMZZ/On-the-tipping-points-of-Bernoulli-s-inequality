% -*- mode: latex; eval: (flyspell-mode 1); ispell-local-dictionary: "american"; TeX-master: t; -*-

\documentclass[12pt]{article}



\usepackage{hyperref,amsthm,amsmath,amsfonts} 

\newcommand{\bZ}{\mathbb{Z}} 
\newcommand{\bR}{\mathbb{R}} 
\newcommand{\bN}{\mathbb{N}}
\newcommand{\gtint}[1]{\left] #1, \infty \right[}
\newcommand{\geint}[1]{\left[ #1, \infty \right[}
\newcommand{\ltint}[1]{\left]- \infty, #1 \right[}
\newcommand{\leint}[1]{\left]- \infty, #1 \right]}

\newcommand{\ttx}{\mathtt{x}}

\newtheorem{theorem}{Theorem}
% \newtheorem{proposition}{Proposition}
\newtheorem{lemma}{Lemma}

\begin{document}

\sloppy

Let $\ttx$ be an indeterminate over $\bR$ and let $n \in \bN$.
Put
$$
f_n(\ttx) = {(1 + \ttx)}^n - 1 - n \ttx \, .
$$
The fundamental property of $f_n$ is:
for each $x \in \bR$, $f_n(x) \ge 0$ is equivalent to ${(1 + x)}^n \ge 1 + n x$; 
the latter inequality is usually called \emph{Bernoulli's inequality}.
For instance, we have 
$f_0(\ttx) = f_1(\ttx) = 0$,
$f_2(\ttx) = \ttx^2$, and
\begin{equation} \label{eq:f3-t3}
f_3(\ttx) = \ttx^3 + 3 \ttx = \ttx^2 (\ttx + 3) \,.
\end{equation}
% and 
% $$
% f_4(\ttx) = \ttx^4 + 4 \ttx^3 + 6 \ttx = \ttx^2 \left( {(\ttx + 2)}^2 + 4 \right)  \, .
% $$
 The following identities are not mentioned anywhere else in the paper but may be of interest to the reader:
 $$
 f_n (\ttx)
 = \ttx^2 \sum_{k = 2}^n \binom{n}{k} \ttx^{k - 2}
 = \ttx^2 \sum_{k = 1}^{n - 1}  \sum_{j = 0}^{k - 1} {(1 + \ttx)}^j \, .  
 $$

\begin{theorem} \label{thm:root-mult}
  For each $n \in \bN + 2$,
  $0$ is a double root of $f_n$ and $0$ is the only non-simple (complex) root of $f_n$.
\end{theorem}

\begin{proof}
  Straightforward computations yield
  \begin{equation} \label{eq:deriv-fn} 
  f_n'(\ttx)  = n \left( {(1 + \ttx)}^{n - 1} -  1 \right) 
  \end{equation}
  and
  $$
  f_n''(\ttx)  = n (n - 1) {(1 + \ttx)}^{n - 2} \,.
  $$
  It follows
  $$
  f_n(0) = f_n'(0) = 0 \ne n (n - 1) = f_n''(0) \, ,
  $$
  whence $0$ is a double root of $f_n$.
  In addition, $f_n$ and its derivative satisfy 
   $$
   (1 + \ttx) f_n'(\ttx) - n f_n(\ttx) = n (n - 1) \ttx \, ,
   $$
   so for every complex number $z$, $f_n(z) = f_n'(z) = 0$ implies $z = 0$.
   It follows that $0$ is the only non-simple root of $f_n$.
 \end{proof}
 
Theorem~\ref{thm:root-mult} ensures that for every $x \in \bR$ and every $n \in \bN + 2$,
the graph of $f_n$ \emph{crosses} the $\ttx$-axis at $\ttx = x$ if, and only if, $x$ is a non-zero root of $f_n$
(the graph of $f_n$ \emph{touches} the $\ttx$-axis at $\ttx = 0$ but does not cross).



 \begin{theorem} \label{thm:variation-even}
   For each $n \in 2 \bN + 2$,
   $f_n$ is
   decreasing on $\leint{0}$ and
   increasing on $\geint{0}$.
 \end{theorem}

 \begin{proof}
   Since $n - 1$ is an odd positive integer,
   Equation~\eqref{eq:deriv-fn} shows that
   $f'_n$ is increasing on $\bR$,
   and subsequently, that
   $0$ is the only real root of $f_n'$.
   Therefore,
   $f'_n$ is negative on $\ltint{0}$
   and 
   positive on $\gtint{0}$.
   The desired result follows.
 \end{proof}


 \begin{theorem} \label{thm:variation-odd}
   For each $n \in 2 \bN + 3$, $f_n$ is
   increasing on $\leint{- 2}$,
   decreasing on $[- 2, 0]$, and
   increasing on $\geint{0}$.
 \end{theorem}

 \begin{proof}
   Since $n - 1$ is an even positive integer,
   the function from $\bR$ to itself associated with the monomial $\ttx^{n - 1}$ is
   decreasing on $\leint{0}$ and
   increasing on $\geint{0}$.
   Hence, 
   Equation~\eqref{eq:deriv-fn} shows: 
   $f_n'$ is decreasing on $\leint{- 1}$,
   $f_n'$ vanishes at $- 2$,
   $f_n'$ is increasing on $\geint{- 1}$, and
   $f_n'$ vanishes at~$0$.
   Therefore, $f_n'$ is positive on $\ltint{- 2}$,
   negative on $\left]- 2, 0 \right[$, and
   positive on $\gtint{0}$.
   The desired result follows.
 \end{proof}

 \begin{theorem}[Bernoulli's inequality]
   \label{thm:Bernoulli}
   For every $n \in \bN + 2$ and every $x \in \bR \setminus \{ 0 \}$ such that $n \in 2 \bN$ or $x \ge - 2$,
   $f_n(x)$ is positive.
 \end{theorem}

 \begin{proof}
   The desired result is an immediate corollary of Theorems \ref{thm:variation-even} and \ref{thm:variation-odd}.
 \end{proof}
 
 
 \begin{theorem} \label{thm:tipping-point}
   For each odd  $n \in 2 \bN + 3$,
   there exists $t_n \in \ltint{- 2}$ such that $f_n$ is 
   negative on $\gtint{t_n}$
   and
   positive on $\left]t_n, 0 \right[ \cup \gtint{0}$.
 \end{theorem} 

 \begin{proof}
   Since $f_n(- 1) = n - 1 > 0$ and since $f_n(x)$ approaches $- \infty$ as $x$ approaches $- \infty$,
   the intermediate value theorem ensures that there exists $t_n \in \leint{- 1}$ such that $f_n(t_n) = 0$.
   Besides, Theorem~\ref{thm:Bernoulli} ensures that
   $f_n$ is positive on $\left[-2, 0 \right[ \cup \gtint{0}$.
   Therefore, $t_n$ belongs to $\ltint{- 2}$ and
   $f_n$ behaves on $\geint{- 2}$ as desired.
   It remains to study the sign of $f_n$ on $\ltint{-2}$.
   Since $f_n$ is increasing on that interval by Theorem~\ref{thm:variation-odd},
   $f_n$ is negative on $\ltint{t_n}$ and
   positive on $\left]t_n, - 2 \right[$.
 \end{proof}

 For each $n \in 2 \bN + 3$,
 let $t_n$ be as in Theorem~\ref{thm:tipping-point}:
 % $t_n \in \ltint{- 2}$,
 % $f_n(t_n) = 0$,
 %${(1 + t_n)}^n = 1 + n t_n$,
% and 
 $$
 \left\{ x \in \bR : f_n(x) \ge 0 \right\}
 %= \left\{ x \in \bR : {(1 + x)}^n \ge 1 + n x  \right\}
 = \geint{t_n} 
 $$
 ($t_n$ is thus a \emph{tipping point} for Bernoulli's inequality).
  
 
 % For instance $t_3  = - 3$.

 \begin{lemma}
   For each $n \in \bN + 2$, $f_n(- 2 - n^{-1})$ is positive.
 \end{lemma}
 
 \begin{proof}
   Theorem~\ref{thm:Bernoulli} ensures $f_n(- {(n + 1)}^{-1}) \ge 0$, whence
   $$
   \left( 1 - \frac{1}{n + 1} \right)^n \ge 1 - \frac{n}{n + 1} = \frac{1}{n + 1} \, .
   $$
   Since 
   $$
   1 + \frac{1}{n} = \left( 1 - \frac{1}{n + 1} \right)^{-1} \,,
   $$
   it follows 
   $$
  \left( 1 + \frac{1}{n} \right)^n = \left( 1 - \frac{1}{n + 1} \right)^{-n}  \le n + 1 
  $$
  (a proof that ${(1 + n^{-1} )}^n$ is less than $e$ requires a little more work \cite{Wiener85}),
  and subsequently,
  $$
  f_n \left(- 2 - \frac{1}{n} \right) = {(- 1)}^n \left( 1 + \frac{1}{n} \right)^n + 2 n \ge - (n + 1) + 2 n = n - 1 > 0 \,.
  $$
 \end{proof} 
 
 \begin{theorem} [N.~Had\v{z}iivanov and I.~Prodanov \cite{MitrinovicAI,MitrinovicCNIA,MondP94,MitrinovicP93}] \label{thm:bulgare}
   The following inequalities hold true for every $n \in 2 \bN + 3$:
   $$
   - 3 \le t_n < t_{n + 2} < - 2 - \frac{1}{n} \,.
   $$
 \end{theorem}
 
 \begin{proof}
   Put $u_n = - 2 - n^{-1}$ for each $n \in \bN{1}$.
   Let us first prove that $f_n$ is positive on $\left]u_{n - 2}, 0 \right[$ for every $n \in \bN{3}$.
   Since $u_3 = - 3$ and $f_3(- 3) = 0$, it suffices to prove that $f_n(u_{n - 2})$ increases with~$n$.
   For each $n \in \bN{1}$, straightforward computations yield 
   $$
   f_{n + 2} (\ttx) = {(1 + \ttx)}^2 f_n(\ttx) + n \ttx^2 (\ttx - u_n) \,.
   $$
   whence
   $$
   f_{n + 2}(u_n)
   =
   {(1 + u_n )}^2 f_n(u_n)
   =
   {\left(1 + \frac{1}{n} \right)}^2 f_n(u_n)
   \ge
   f_n(u_n)
   >
   f_n(u_{n - 2})
   $$
   because $u_{n - 2} < u_n < - 2$ and $f_n$ is increasing 
   and
   $$
   0 = {(1 + t_{n + 2})}^2 f_n(t_{n + 2}) +  n t_{n + 2}^2 (t_{n + 2} - u_n)  
   $$
   if $n$ is odd.
 \end{proof}   

 \bibliographystyle{plain}
  \bibliography{bernoulli}

\end{document}
