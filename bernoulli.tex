% -*- mode: latex; eval: (flyspell-mode 1); ispell-local-dictionary: "american"; TeX-master: t; -*-

\documentclass[12pt]{article}



\usepackage{hyperref,amsthm,amsmath,amsfonts} 

\newcommand{\bZ}{\mathbb{Z}} 
\newcommand{\bR}{\mathbb{R}} 
\newcommand{\bN}{\mathbb{N}}
\newcommand{\abs}[1]{\left| #1 \right|}
\newcommand{\gtint}[1]{\left] #1, \infty \right[}
\newcommand{\geint}[1]{\left[ #1, \infty \right[}
\newcommand{\ltint}[1]{\left]- \infty, #1 \right[}
\newcommand{\leint}[1]{\left]- \infty, #1 \right]}

\newcommand{\ttx}{\mathtt{x}}
\newcommand{\tty}{\mathtt{y}}

\newtheorem{theorem}{Theorem}
% \newtheorem{proposition}{Proposition}
\newtheorem{lemma}{Lemma}

\begin{document}

\title{On the tipping points of Bernoulli's inequality}
\author{Charlot Colmes}
\maketitle 

\sloppy

Let $\bR$ denote the field of real numbers,
let $\bN$ denote the set of all \emph{non-negative} integers, and 
let $\ttx$ be an indeterminate over~$\bR$.

\section{Introduction}

The paper focuses on the sequence  of polynomials $\left( f_n \right)_{n \in \bN}$ given by:
\begin{equation} \label{eq:def-fn}
f_n(\ttx) = {(1 + \ttx)}^n - 1 - n \ttx  
\end{equation}
for each $n \in \bN$.
The first five terms are:
$f_0(\ttx) = f_1(\ttx) = 0$,
$f_2(\ttx) = \ttx^2$,
$f_3(\ttx) = \ttx^3 + 3 \ttx^2$, and
$f_4(\ttx) = \ttx^4 + 4 \ttx^3 + 6 \ttx^2$.
The fundamental property of the $f_n$'s is that
the following equivalence holds true for every $n \in \bN$ and every $x \in \bR$:
$$
f_n(x) \ge 0 \iff {(1 + x)}^n \ge 1 + n x \,.
$$ 
The latter inequality is called \emph{Bernoulli's inequality}.
Hence, studying Bernoulli's inequality is studying the signs of the $f_n$'s.
It turns out that the behavior of $f_n$ depends on the parity of $n$:


\begin{theorem} \label{thm:variation}
   For each $n \in 2 \bN + 2$,
   $f_n'$ is
   negative on $\ltint{0}$ and
   positive on $\gtint{0}$.
   For each $n \in 2 \bN + 3$, $f_n'$ is
   positive on $\ltint{- 2}$,
   negative on $\left]- 2, 0 \right[$, and
   positive on $\gtint{0}$.
 \end{theorem}

 \begin{proof}
   The basic rules of differentiation yield 
\begin{equation} \label{eq:deriv-fn} 
  f_n'(\ttx)  = n \left( {(1 + \ttx)}^{n - 1} -  1 \right) 
\end{equation}
for each $n \in \bN$,
and thus, the desired result follows from a simple examination of the latter equation.
%Equation~\eqref{eq:deriv-fn}.
\end{proof} 


 \begin{theorem} \label{thm:Bernoulli}
   For each $n \in 2 \bN + 2$,
   $f_n$ is positive on $\bR \setminus \{ 0 \}$. 
   For each  $n \in 2 \bN + 3$,
   there exists $t_n \in  \ltint{-2}$
   such that $f_n$ is
   negative on $\ltint{t_n}$ and
   positive on $\left]t_n,  0\right[ \cup \gtint{0}$.
 \end{theorem}
 
 \begin{proof}
   First, the desired result holds true for $n \in \{ 0, 1 \}$ because $f_0(\ttx) = f_1(\ttx) = 0$.
   Second, assume $n \in  2 \bN + 2$.
   It then follows from Theorem~\ref{thm:variation} that 
   $f_n$ is decreasing on $\leint{0}$ and increasing on $\geint{0}$.
   Therefore, $f_n$ attains its minimum at $0$ and nowhere else. 
   Since $f_n(0) = 0$, $f_n$ is positive on $\bR \setminus \{ 0 \}$.
   Third and last, assume $n \in 2 \bN + 3$.
   It then follows from Theorem~\ref{thm:variation} that
   the restriction of $f_n$ to $\geint{-2}$ attains its minimum at $0$
   and nowhere else.
   Therefore, $f_n$ is positive on $\left[- 2, 0\right[ \cup \geint{0}$.  
   Moreover,
   $f_n(x)$ approaches $- \infty$ as $x$ approaches $- \infty$
   because $f_n$ is a polynomial of odd degree~$n$.
   Hence,
   the intermediate value theorem ensures that there exists $t_n \in \ltint{- 2}$ such that $f_n(t_n) = 0$.
   Since $f_n$ is increasing on $\leint{- 2}$ by Theorem~\ref{thm:variation},
   $f_n$ is negative on $\ltint{t_n}$ and positive on $\left]t_n, - 2 \right]$.
   It follows that $f_n$ is positive on
   $\left]t_n, - 2 \right] \cup \left[- 2, 0 \right[ \cup \gtint{0}
   =
   \left]t_n, 0 \right[ \cup \gtint{0}$,
   as desired.
 \end{proof}
 
 The $t_n$'s are, by definition, the \emph{tipping points} for Bernoulli's inequality.
 Its first term is easy to compute:
 since $f_3$ factors into $f_3(\ttx) = \ttx^2( \ttx + 3)$,
 we have $t_3 = - 3$.
 The aim of this paper is to study the sequence $\left( t_n \right)_{n \in 2 \bN + 3}$.
 Previously known results are examined in Section~\ref{sec:bulgar}:
 % we show that our sequence is increasing
 % and
 % $t_n$ is less than $- 2 - n^{-1}$ for every $n \in 2 \bN + 3$.
 Some improvements are presented in Section~\ref{sec:contrib}.
 % we show $t_n \le - 2 - \ln(2 n) n^{-1}$ is asymptotically equivalent to $(\ln n) / n$ as $n$ approaches~$\infty$.

 
 
 \section{Previous results} \label{sec:bulgar} 

 

 Equation~\eqref{eq:f3-t3} shows that $t_3 = - 3$. 
 \begin{equation} \label{eq:f3-t3}
f_3(\ttx) = \ttx^3 + 3 \ttx^2 = \ttx^2 (\ttx + 3) \,.
\end{equation}



 



 \section{Previous results} 

 In the literature \cite{MitrinovicCNIA, MitrinovicAI, MitrinovicP93, MondP94},
 both Theorems \ref{thm:HP-upper} and \ref{thm:HP-tn-increase} below are attributed to N.~Had\v{z}iivanov and I.~Prodanov.
 
 Straightforward computations yield the following recurrence relation which is used as a lemma:
 \begin{equation} \label{eq:fn+2-fn}
   f_{n + 2} (\ttx) = {(1 + \ttx)}^2 f_n(\ttx) + n \ttx^2 \left(\ttx + 2 + \frac{1}{n} \right)
 \end{equation}
 for every $n \in \bN + 1$.
 
 \begin{theorem} \label{thm:HP-upper}
   For each $n \in 2 \bN + 3$, $t_n$ is not greater than $- 2 - {(n - 2)}^{-1}$.
 \end{theorem}

 \begin{proof}
   Put $u_n = - 2 - n^{-1}$ for every $n \in \bN + 1$.
   Our task it to prove $t_n \le u_{n - 2}$ for every $n \in 2 \bN + 3$.
   We proceed by induction on~$n$.
   Since $f_3(- 3) = 0$,
   we have $t_3 = - 3 = u_3$, 
   
   The assertion holds true for $n = 3$ because $t_3 = u_1 = - 3$.
   Let us now deal with the inductive step.
   Let $n \in 2 \bN + 3$ be such that $t_n \le u_{n - 2}$.
   Since $u_{n - 2}$ not greater than $u_n$, $t_n$ is not greater than $u_n$, or, equivalently, $f_n(u_n)$ is non-negative.
   Therefore,  
   by letting $\ttx = u_n$ in Equation~\eqref{eq:fn+2-fn}, we obtain (note that the rightmost term vanishes)
   $f_{n + 2}(u_n) = {( 1 + u_n )}^2 f_n(u_n) \ge 0$.
   It follows $t_{n + 2} \le u_n$, as desired.
 \end{proof}
 
 
 
\begin{theorem} \label{thm:HP-tn-increase} 
  The sequence $\left( t_n \right)_{n \in 2 \bN + 3}$ is increasing.
 \end{theorem} 

 \begin{proof}
 Let $n \in 2 \bN + 3$.
 By letting $\ttx = t_n$ in Equation~\eqref{eq:fn+2-fn}, we obtain
 $$ 
f_{n + 2} (t_n) = n t_n^2 \left(t_n - 2 - \frac{1}{n} \right) \, .
$$
Besides, Theorem~\ref{thm:HP-upper} ensures that the right-hand side of the latter equality is negative.
It follows $f_{n + 2} (t_n) < 0$ or, equivalently, $t_n < t_{n + 2}$.
\end{proof}

For each $n \in 2 \bN + 5$, $t_n$ lies in $\left]- 3, - 2 \right[$ and the coefficients of $f_n$ are all integers,
so the integral root theorem ensures that $t_n$ is irrational.

\section{Our contribution}

\begin{theorem}
  Let $a \in \gtint{0}$ and let $b \in \bR$.
 As $x$ approaches $\infty$, 
 ${(a x + b)}^{1 / x} - 1$
 and
 $x^{-1} \ln x$ are asymptotically equivalent.
\end{theorem}

\begin{proof}
  
  $$
  {(a x + b)}^{1 / x} - 1 = \exp \left( \frac{\ln (a x + b)}{x} \right) - 1 \ge   \frac{\ln (a x + b)}{x} 
  $$
  
  $$
  \exp(y) = 1 + y + \frac{1}{2} y^2 + O(y^3)  
  $$

  $$
  \ln(1 + y) = O(y) 
  $$

  $$
  \ln(a x + b) = \ln (a x) + \ln \left(1 +  \frac{b}{a x} \right) = \ln(a x) + O \left( \frac{1} {x} \right)
  $$

  $$
   \frac{\ln(ax + b)}{x} = \frac{\ln(a x)}{x} + O \left( \frac{1}{x^2} \right)
  $$
  $$
  {(a x + b)}^{1 / x} = \exp \left( \frac{\ln( a x + b)}{x}  \right) = 1 + 
  $$
\end{proof} 

\begin{proof}
  Put $\epsilon_n = - t_n - 2$: $t_n = - 2 - \epsilon_n$.
  Equality
  $$
  {(1 + t_n)}^n = 1 + n t_n 
  $$
  $$
   {(1 + \epsilon_n)}^n  = - 1 + 2 n + n \epsilon_n 
   $$

   $$
   2 n - 1 \le {(1 + \epsilon_n)}^n  \le  2 n 
   $$

   $$
   \sqrt[n]{2 n - 1} - 1 \le \epsilon_n  \le \sqrt[n]{2 n} - 1
   $$

   $$
   \exp(x) - 1 \ge x
   $$
   $$
   \sqrt[n]{2n - 1} - 1 \ge \frac{\ln(2n - 1)}{n} \sim \frac{\ln(2n)}{n} 
   $$

   $$
   \exp(x) - 1 \sim  x
   $$
   
   $$
   \sqrt[n]{2n \pm 1} - 1 \sim \frac{\ln(2n \pm 1)}{n} \sim \frac{\ln(2n)}{n} 
   $$
   
\end{proof}
Let $n \in \bN$.
Straightforward computations yield
$$
f_n(- 2 - \ttx) = {(- 1)}^n {(1 + \ttx)}^n - 1 + 2n + n \ttx \, , 
$$
whence 
\begin{equation} \label{eq:fn-2-x-odd}
  n \in 2 \bN + 1
  \implies 
 f_n(- 2 - \ttx) = 2n - 2 - f_n(\ttx)   \, .
\end{equation} 
 
\begin{theorem} \label{thm:lower-sqrt}
  For every $n \in 2 \bN + 3$, $t_n$ is not less than $- 2 - 2 / \sqrt{n}$.
\end{theorem}

\begin{proof}
  Let $n \in 2 \bN + 3$.
 Put $\epsilon_n = 2 / \sqrt{n}$.
  Since $\epsilon_n \ge 0$,
  the binomial theorem yields
  $$
  f_n(\epsilon_n ) \ge \frac{1}{2} (n - 1) n \epsilon_n^2 = 2 n - 2 \,.
  $$
  It follows $f_n(- 2 - \epsilon_n) \le 0$ by Equation~\eqref{eq:fn-2-x-odd}, whence $t_n \ge - 2 - \epsilon_n$.
\end{proof} 

Note that Theorem~\ref{thm:lower-sqrt} ensures that $t_n$ approaches $- 2$ as $n$ approaches~$\infty$.


 \bibliographystyle{plain}
 \bibliography{bernoulli}

 \section{Bonus section}
 
\begin{theorem} \label{thm:root-mult}
  For each $n \in \bN + 2$,
  $0$ is a double root of $f_n$ and $0$ the only non-simple (complex) root of $f_n$.
\end{theorem}

\begin{proof}
  Straightforward computations yield the second derivative of $f_n$:
\begin{equation} \label{eq:deriv-second-fn}
f_n''(\ttx)  = n (n - 1) {(1 + \ttx)}^{n - 2} \, .
\end{equation}
  Now, by letting $\ttx = 0$ in Equations \eqref{eq:def-fn}, \eqref{eq:deriv-fn}, and \eqref{eq:deriv-second-fn},
  we obtain 
  $$
  f_n(0) = f_n'(0) = 0 \ne n (n - 1) = f_n''(0) \, ,
  $$
  whence $0$ is a double root of $f_n$.
  In addition, $f_n$ and its derivative satisfy 
   $$
   (1 + \ttx) f_n'(\ttx) - n f_n(\ttx) = n (n - 1) \ttx \, ,
   $$
   so for every complex number $z$, $f_n(z) = f_n'(z) = 0$ implies $z = 0$.
   It follows that $0$ is the only non-simple root of $f_n$.
 \end{proof}
 
%  Theorem~\ref{thm:root-mult} ensures that
% for every $n \in \bN + 2$ and every $x \in \bR$, 
% the graph of $f_n$ \emph{crosses} the $\ttx$-axis at $\ttx = x$ if, and only if, $x$ is a non-zero root of $f_n$
% (the graph of $f_n$ \emph{touches} the $\ttx$-axis at $\ttx = 0$ but does not cross).

\begin{theorem} \label{thm:increasing-fn}
  For each $x \in \left]- 2, 0 \right[ \cup \gtint{0}$,
  the sequence $\left( f_n(x) \right)_{n \in \bN + 1}$ is increasing.
  For each $x \in \bR \setminus \{ 0 \}$,
  the sequence $\left( f_n(x)\right)_{n \in 2 \bN}$ is increasing.
\end{theorem} 

\begin{proof}
 Let $n \in \bN + 1$.
 Straightforward computations yield  
 $$
 f_{n+ 1} (\ttx) - f_n(\ttx) = \frac{1}{n + 1} \ttx  f_{n + 1}'(\ttx) 
 $$
 and
 Theorem~\ref{thm:variation} ensures that $\ttx f_{n + 1}'(\ttx)$ is positive on $\left]- 2, 0 \right[ \cup \gtint{0}$.
 Therefore, the first part of the desired result holds true.
 Now, let $n \in 2 \bN$ and let $x \in \leint{-2}$.
 Straightforward computations yield  
 $$
 f_{n + 2}(\ttx) - f_n(\ttx) =  \left(  (2 + \ttx) {(1 + \ttx)}^n - 2 \right) \ttx
 $$
 and, besides, we have
 $$
  (2 + x) {(1 + x)}^n \le 0 \,, 
 $$
 whence $f_{n + 2}(x) - f_n(x) \ge - 2 x   \ge 4$.
 Therefore, the second part of the desired result holds true.
\end{proof} 

\end{document}

 \begin{theorem}%[Bernoulli's inequality]
   \label{thm:Bernoulli}
   For each $n \in 2 \bN + 2$,
   $f_n(x)$ is positive on $\bR \setminus \{ 0 \}$. 
   For each  $n \in 2 \bN + 3$,
   there exists $t_n \in  \ltint{-2}$
   such that $f_n$ is negative on $\ltint{t_n}$ and positive on $\left]t_n, 0 \right[ \cup \gtint{0}$.
 \end{theorem}
 
 \begin{proof}
   First, assume $n \in  2 \bN + 2$.
   It then follows from Theorem~\ref{thm:variation} that 
   $f_n$ is decreasing on $\leint{0}$ and increasing on $\geint{0}$.
   Therefore, $f_n$ attains its minimum at~$0$ and nowhere else.
   Since $f_n(0) = 0$, $f_n$ is positive on $\bR \setminus \{ 0 \}$.
   Now, assume $n \in 2 \bN + 3$.
   It then follows from Theorem~\ref{thm:variation} that
   the restriction of $f_n$ to $\geint{-2}$ attains its minimum at $0$
   and nowhere else.
   Therefore, $f_n$ is positive on $\left[- 2, 0 \right[ \cup \gtint{0}$.
   Moreover,
   $f_n$ is a polynomial of odd degree $n$, so $f_n(x)$ approaches $- \infty$ as $x$ approaches $- \infty$.
   Hence,
   the intermediate value theorem ensures that there exists $t_n \in \ltint{- 2}$ such that $f_n(t_n) = 0$.
   Since $f_n$ is increasing on $\leint{- 2}$ by Theorem~\ref{thm:variation},
   $f_n$ is negative on $\ltint{t_n}$ and positive on $\left]t_n, - 2 \right]$.
   It follows that $f_n$ is positive on
   $\left]t_n, - 2 \right] \cup \left[- 2, 0 \right[ \cup \gtint{0} = \left]t_n, 0 \right[ \cup \gtint{0}$, as desired.
 \end{proof}