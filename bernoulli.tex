% -*- mode: latex; eval: (flyspell-mode 1); ispell-local-dictionary: "american"; TeX-master: t; -*-

\documentclass[12pt]{article}



\usepackage{hyperref,amsthm,amsmath,amsfonts} 

\newcommand{\bZ}{\mathbb{Z}} 
\newcommand{\bR}{\mathbb{R}} 
\newcommand{\bN}[1]{\bZ_{\ge #1}} 
\newcommand{\gtint}[1]{\left] #1, \infty \right[}
\newcommand{\geint}[1]{\left[ #1, \infty \right[}
\newcommand{\ltint}[1]{\left]- \infty, #1 \right[}
\newcommand{\leint}[1]{\left]- \infty, #1 \right]}
% \newcommand{\Rneg}{\left]- \infty, 0 \right[}
 %\newcommand{\Rnpos}{\left]- \infty, 0 \right]}
% \newcommand{\Rnneg}{\left[0, \infty\right[}
% \newcommand{\Rpos}{\left]0, \infty\right[}

\newcommand{\ttx}{\mathtt{x}}

\newtheorem{theorem}{Theorem}
% \newtheorem{proposition}{Proposition}
% \newtheorem{lemma}{Lemma}

\begin{document}

\sloppy

Let $\ttx$ be an indeterminate over $\bR$ and let $n \in \bN{0}$.
Put
$$
f_n(\ttx) = {(1 + \ttx)}^n - 1 - n \ttx \, .
$$
For each $x \in \bR$, $f_n(x) \ge 0$ is equivalent to ${(1 + x)}^n \ge 1 + n x$; 
the latter inequality is usually called \emph{Bernoulli's inequality}.
%Note  $f_n$ vanishes at~$0$. 
For instance, we have 
$f_0(\ttx) = f_1(\ttx) = 0$,
$f_2(\ttx) = \ttx^2$,
$$
f_3(\ttx) = \ttx^3 + 3 \ttx = \ttx^2 (\ttx + 3) \,,
$$
and 
$$
f_4(\ttx) = \ttx^4 + 4 \ttx^3 + 6 \ttx = \ttx^2 \left( {(\ttx + 2)}^2 + 4 \right)  \, .
$$

\begin{theorem} \label{thm:root-mult}
  For each $n \in \bN{2}$,
  $0$ is a double root of $f_n$ and $0$ is the only non-simple (complex) root of $f_n$.
\end{theorem}

\begin{proof}
  Straightforward computations yield
  \begin{equation} \label{eq:deriv-fn} 
  f_n'(\ttx)  = n \left( {(1 + \ttx)}^{n - 1} -  1 \right) 
  \end{equation}
  and
  $$
  f_n''(\ttx)  = n (n - 1) {(1 + \ttx)}^{n - 2} \,.
  $$
  It follows
  $$
  f_n(0) = f_n'(0) = 0 \ne n (n - 1) = f_n''(0) \, ,
  $$
  whence $0$ is a double root of $f_n$.
  In addition, $f_n$ and its derivative satisfy 
   $$
   (1 + \ttx) f_n'(\ttx) - n f_n(\ttx) = n (n - 1) \ttx \, ,
   $$
   so for every complex number $z$, $f_n(z) = f_n'(z) = 0$ implies $z = 0$.
   It follows that $0$ is the only non-simple root of $f_n$.
 \end{proof}
 
Theorem~\ref{thm:root-mult} ensures that for every $x \in \bR$ and every integer $n \in \bN{2}$,
the graph of $f_n$ \emph{crosses} the $\ttx$-axis at $\ttx = x$ if, and only if, $x$ is a non-zero root of $f_n$
(the graph of $f_n$ \emph{touches} the $\ttx$-axis at $\ttx = 0$ but does not cross it).
 The following identities hold for every $n \in \bN{0}$ and may be of interest to the reader:
 $$
 f_n (\ttx)
 = \ttx^2 \sum_{k = 2}^n \binom{n}{k} \ttx^{k - 2}
 = \ttx^2 \sum_{k = 1}^{n - 1}  \sum_{j = 0}^{k - 1} {(1 + \ttx)}^j \, .  
 $$



 \begin{theorem} \label{thm:n-is-even}
   For each even  $n \in \bN{2}$,
   $f_n$ is
   decreasing on $\leint{0}$,
   increasing on $\geint{0}$, and
   positive on $\bR \setminus \{ 0 \}$.
 \end{theorem}

 \begin{proof}
   Since $n - 1$ is an odd positive integer,
   Equation~\eqref{eq:deriv-fn} shows that
   $f'_n$ is increasing on $\bR$,
   and subsequently, that
   $0$ is the only real root of $f_n'$.
   Therefore,
   $f'_n$ is negative on $\ltint{0}$
   and 
   positive on $\gtint{0}$.
   The desired result follows.
 \end{proof}


 \begin{theorem} \label{thm:variation-odd}
   For each odd $n \in \bN{3}$, $f_n$ is
   increasing on $\leint{- 2}$,
   decreasing on $[- 2, 0]$, and
   increasing on $\geint{0}$.
 \end{theorem}

 \begin{proof}
   Since $n - 1$ is an even positive integer,
   the polynomial function from $\bR$ to itself associated with $\ttx^{n - 1}$ is decreasing on $\leint{0}$
   and increasing on $\geint{0}$.
   Hence, 
   Equation~\eqref{eq:deriv-fn} shows: 
   $f_n'$ is decreasing on $\leint{- 1}$,
   $f_n'$ vanishes at $- 2$,
   $f_n'$ is increasing on $\geint{- 1}$, and
   $f_n'$ vanishes at~$0$.
   Therefore, $f_n'$ is positive on $\ltint{- 2}$,
   negative on $\left]- 2, 0 \right[$, and
   positive on $\gtint{0}$.
   The desired result follows.
 \end{proof}
 
 \begin{theorem} \label{thm:tipping-point}
   For each odd  $n \in \bN{3}$,
   there exists $t_n \in \ltint{- 2}$ such that $f_n$ is 
   negative on $\left]- \infty, t_n \right[$
   and
   positive on $\left]t_n, 0 \right[ \cup \gtint{0}$.
 \end{theorem} 

 \begin{proof}
   Since $f_n(- 1) = n - 1 > 0$ and since $f_n(x)$ approaches $- \infty$ as $x$ approaches $- \infty$,
   the intermediate value theorem ensures that there exists $t_n \in \leint{- 1}$ such that $f_n(t_n) = 0$.
   Besides, it follows from Theorem~\ref{thm:variation-odd} that $f_n$ is larger than $f_n(0) = 0$  on
   $\left[- 2, 0 \right[ \cup \gtint{0}$.
   Therefore, $t_n$ belongs to $\ltint{- 2}$ and
   $f_n$ behaves on $\geint{- 2}$ as desired.
   It remains to study $f_n$ on $\ltint{-2}$.
   By Theorem~\ref{thm:variation-odd},
   $f_n$ is increasing on that interval,
   and thus $f_n$ is negative on $\ltint{t_n}$ and
   positive on $\left]t_n, - 2 \right[$.
 \end{proof}

 For each odd $n \in \bN{3}$,
 let $t_n$ be as in Theorem~\ref{thm:tipping-point}:
 $t_n \in \left]- \infty, - 2 \right[$, $f_n(t_n) = 0$, ${(1 + t_n)}^n = 1 + n t_n$,
 and 
 $$
 \left\{ x \in \bR : f_n(x) \ge 0 \right\}
 = \left\{ x \in \bR : {(1 + x)}^n \ge 1 + n x  \right\}
 = \left[t_n, \infty \right[ 
 $$
 ($t_n$ is thus a \emph{tipping point} for Bernoulli's inequality).
  
 
 % For instance $t_3  = - 3$.


 \begin{theorem} [Had\v{z}iivanov and Prodanov \cite{MitrinovicAI,MitrinovicCNIA}] \label{thm:bulgare}
   The following inequalities hold true for every odd $n \in \bN{3}$:
   $$ - 3 \le t_n < t_{n + 2} < - 2 - \frac{1}{n} \,.
   $$
 \end{theorem}
 
 \begin{proof}
   Put $u_n = - 2 - n^{-1}$ for each $n \in \bN{1}$.
   Let us first prove that $f_n$ is positive on $\left]u_{n - 2}, 0 \right[$ for every $n \in \bN{3}$.
   Since $u_3 = - 3$ and $f_3(- 3) = 0$, it suffices to prove that $f_n(u_{n - 2})$ increases with~$n$.
   For each $n \in \bN{1}$, straightforward computations yield 
   $$
   f_{n + 2} (\ttx) = {(1 + \ttx)}^2 f_n(\ttx) + n \ttx^2 (\ttx - u_n) \,.
   $$
   whence
   $$
   f_{n + 2}(u_n)
   =
   {(1 + u_n )}^2 f_n(u_n)
   =
   {\left(1 + \frac{1}{n} \right)}^2 f_n(u_n)
   \ge
   f_n(u_n)
   >
   f_n(u_{n - 2})
   $$
   because $u_{n - 2} < u_n < - 2$ and $f_n$ is increasing 
   and
   $$
   0 = {(1 + t_{n + 2})}^2 f_n(t_{n + 2}) +  n t_{n + 2}^2 (t_{n + 2} - u_n)  
   $$
   if $n$ is odd.
   
   % Since $u_n \ne - 1$, Equation~\eqref{ } yield
   % $$

   % $$
   % Since $t_n \notin \{ - 1, 0 \}$, 
   % The latter two equalities yield
   % $$
   % f_{n + 2}(u_n) = f_n(u_n)
   % $$
   % and
   % $$
   
   % $$
   % and thar have equal sign and
   
 for every $n \in \bN{3}$ and
 $$
 f_n(t_{n + 2})  (u(n) - t_{n + 2})  \ge 0
 $$
 for every odd $n \in \bN{3}$.
 
   Let us first prove by induction on $n$ that $t_n$ is not greater than $a(n + 2)$ for every odd $n \in \bN{3}$.
   
 \end{proof}
 \begin{theorem}
   \label{lem:odd-two-roots}
   % For each odd $n \in \bN{3}$,  $- 3 \le t_n < t_{n + 2} < - 2

   % $$f_n$ is positive on $\left]- 2 - {(n - 2)}^{- 1}, - 2 \right[$.
 \end{theorem}

 \begin{proof}
   We proceed by induction on~$n$.
   For each integer $n \ge 3$, put
   $$
   a_n = - 2 - \frac{1}{n - 2} 
   $$
   and
   $$
   I_n = \left] a_n, 0 \right[ \cup \left]0, \infty \right[ \, .
   $$
   Since $a_3 = - 3$ and since $f_3(\ttx) = \ttx^2 (x + 3)$,
   $f_3$ is positive on $I_3$;
   since
   $$
   f_4(\ttx) = \ttx^2 ({(\ttx + 2)}^2 + 2 ) \,, 
   $$
   $f_4$ is positive on $I_4$. 
   Hence, the basis of our induction holds.
   Let us now prove the induction step.
   It  relies on the following recurrence relation which holds for every integer $n \ge 3$:
   Assume that $f_n$ is positive on $I_n$ for some integer $n \ge 3$.
   Let $x \in I_{n + 2}$.
   Since $I_{n + 2} \subseteq I_n$,
   $f_n(x)$ is positive.
   Moreover, 
   $n x^2(x - a_{n + 2})$ is positive and ${(1 + x)}^2$ is non-negative.
   It thus follows from Equation~\eqref{eq:recur-f-n+2} that $f_n(x)$ is positive.
   Hence, $f_{n + 2}$ is positive on $I_{n + 2}$.
  \end{proof} 
 % It remains to prove that $f_n$ has a root in $\left]- \infty, a_n \right]$,
 %   where $a_n = - 2 - {(n - 2)}^{-1}$.
 %   Since $f_n(x)$ approaches $- \infty$ as $x$ approaches $- \infty$,
 %   it suffices to check that $f_n(a_n)$ is non-negative.
 %   We prove by induction on $n$ then $f_n$ is non-negative on $\left[a_n, \infty \right[$.
 %   Since $f_3(\ttx) = \ttx^2(\ttx + 3)$ ans since $a_3 = - 3$,
 %   the basis of our induction holds true.
   % Let us now assume $f_n(\ttx)$ is non-negative on $[a_n, - 2]$ for some odd integer $n \ge 3$.
   % In particular, $f_n(\ttx)$ is non-negative on $[a_{n + 2}, - 2]$.
 
   % whence
   %  $$
   %  f_{n + 2}(a_{n + 2}) = {(1 + a_{n + 2})}^2 f_n(a_{n + 2}) \, .
   %  $$
   %  Since $a_n < a_{n + 2}  < 0$,
   %  $f_n$ has no roots in $\left]a_n,  0 \right[$.
   %  Besides, 
    
%    %\end{proof}
  
% \begin{theorem} \label{lem:even-one-root}
%   For each even integer $n \ge 2$,
%   $f_n$ is positive on $\left]- \infty, 0 \right[ \cup \left]0, \infty \right[$.
%  \end{theorem} 

%  \begin{proof}
%    Since $f_2(\ttx) = \ttx^2$, we
%  \end{proof} 

%  \begin{theorem}[Bernoulli's inequality] \label{thm:Bernoulli}
%    Let $n$ be an integer greater than $1$ and let $x$ be a non-zero real number.
%    If $n$ is even or if $x \ge - 2$ then $f_n(x)$ is positive.
%  \end{theorem}
 
 
%  \begin{proof}
%    First, Lemmas \ref{lem:even-one-root} and  \ref{lem:odd-two-roots} ensure that
%    $f_n$ has not root in  $\left]0, \infty \right[$.   
%    Besides, $f_n(t)$ approaches $\infty$ as $t$ approaches $\infty$.
%    Therefore, $f_n$ is positive on $\left]0, \infty \right[$.
%    Second, assume that $n$ is even.
%    Then, Lemma~\ref{lem:even-one-root} ensures that $f_n$ has no negative roots.
%    Besides,  $f_n(t)$ approaches $\infty$ as $t$ approaches $- \infty$.
%    Therefore, $f_n$ is positive on $\left]- \infty, 0 \right[$.
%    Third, assume that $n$ is odd.
%    Lemma~\ref{lem:even-one-root} ensures that $f_n$ has no root in $\left[- 2, 0 \right[$.
%    Besides, $f(- 1) = n - 1$ is positive.
%    Therefore, $f_n$ is positive on $\left[- 2, 0 \right[$.
%  \end{proof}

%  Let $n$ be an odd integer greater than~$1$.
%  Lemma~\ref{lem:odd-two-roots}, there exists $t_n \in \left]- \infty, - 2 \right]$
%  Let $t_n$ denote the unique non-zero real root of $f_n$
  \bibliographystyle{plain}
  \bibliography{bernoulli}

\end{document}
